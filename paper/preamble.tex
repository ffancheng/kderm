%% Any special functions or other packages can be loaded here.
\usepackage[utf8]{inputenc}
\usepackage[flushleft]{threeparttable}
\usepackage{multirow}
\usepackage[normalem]{ulem}
\useunder{\uline}{\ul}{}
%% \usepackage{makecell,interfaces-makecell} %latex error
\usepackage[]{graphicx}
\usepackage[]{color}
\usepackage{xcolor}
% \usepackage[ruled,vlined,linesnumbered]{algorithm2e}
\usepackage{algorithmic}
\usepackage{bm}
\usepackage{etoolbox}
\usepackage{tabu}
\usepackage{placeins}
\setlength {\marginparwidth }{2cm}
\usepackage{todonotes}
\urlstyle{tt}  % don't use monospace font for urls
\allowdisplaybreaks


\usepackage[ruled,vlined,linesnumbered]{algorithm2e}
\usepackage{amsmath, amsfonts, amssymb, bm, mathtools, todonotes, booktabs, microtype}

\makeatletter
%If bera loaded, this is monash wp.
\@ifpackageloaded{bera}{}{\usepackage{mathptmx}}
\makeatother
\usepackage[cal = txupr]{mathalpha}


\makeatletter
\@ifpackageloaded{geometry}{}{\usepackage{geometry}}
\makeatother
\geometry{margin=2.2cm}

\mathtoolsset{showonlyrefs}
\newcommand{\argmax}{\mathop{\text{arg\,max}}}
\clubpenalty = 10000
\widowpenalty = 10000
\brokenpenalty = 10000
\usepackage{graphicx}
\setcounter{topnumber}{2}
\setcounter{bottomnumber}{2}
\setcounter{totalnumber}{4}
\renewcommand{\topfraction}{0.9}
\renewcommand{\bottomfraction}{0.9}
\renewcommand{\textfraction}{0.1}
\renewcommand{\floatpagefraction}{0.9}

\usepackage{caption}
\DeclareCaptionStyle{italic}[justification=centering]
 {labelfont={bf},textfont={it},labelsep=colon}
\captionsetup[figure]{style=italic,format=hang,singlelinecheck=true}
\captionsetup[table]{style=italic,format=hang,singlelinecheck=true}



\usepackage{booktabs}
\usepackage{longtable}
\usepackage{array}
\usepackage{multirow}
\usepackage{wrapfig}
\usepackage{float}
\usepackage{colortbl}
\usepackage{pdflscape}
\usepackage{tabu}
\usepackage{threeparttable}
\usepackage{threeparttablex}
\usepackage[normalem]{ulem}
\usepackage{makecell}


% theorem in latex
\usepackage{amsthm}
\newtheorem{theorem}{Theorem}[section]
\newtheorem{lemma}{Lemma}[section]
\newtheorem{corollary}{Corollary}[section]
\newtheorem{proposition}{Proposition}[section]
\newtheorem{conjecture}{Conjecture}[section]
\theoremstyle{definition}
\newtheorem{definition}{Definition}[section]
\theoremstyle{definition}
\newtheorem{example}{Example}[section]
\theoremstyle{definition}
\newtheorem{exercise}{Exercise}[section]
\theoremstyle{definition}
\newtheorem{hypothesis}{Hypothesis}[section]
\theoremstyle{remark}
\newtheorem*{remark}{Remark}
\newtheorem*{solution}{Solution}
