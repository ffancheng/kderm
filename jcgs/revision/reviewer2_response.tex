\documentclass{article}
\usepackage{geometry} % Set margins
\usepackage{setspace} % For double spacing
\usepackage{titlesec} % For section formatting
\usepackage{enumitem} % For list formatting
\usepackage{xcolor} % For highlighting text
\usepackage{amssymb,tabu,placeins,algorithmic} % natbib
\usepackage{mathtools,bm}
\usepackage[ruled,vlined,linesnumbered]{algorithm2e}
% \usepackage{natbib}
\usepackage{lipsum}
\usepackage[style=authoryear]{biblatex}
\addbibresource{../kderm.bib}
\usepackage{hyperref}

\renewcommand*{\finalnamedelim}{\addspace\&\space}
\renewcommand*{\finallistdelim}{\addspace\&\space}
\renewcommand*{\multicitedelim}{\addsemicolon\space}

% Set page margins
\geometry{
    left=1in,
    right=1in,
    top=1in,
    bottom=1in
}

% Set section formatting
\titleformat{\section}
{\bfseries\Large}
{\thesection.}
{0.5em}
{}

% Set list formatting
\setlist{noitemsep}

% Set double spacing
\doublespacing
\setlength{\parskip}{1em}

% Define colors for highlighting
\definecolor{myblue}{RGB}{0,0,200}
\definecolor{myblack}{RGB}{0, 0, 0}

% Color algorithm
\newenvironment{algocolor}{%
   \setlength{\parindent}{0pt}
   \itshape
   \color{myblue}
}{}

\date{\vspace{-8ex}}

\begin{document}

\title{Response to Reviewer 2's \\ \textit{"Report on 'Distortion corrected kernel density estimator on Riemannian manifolds' by F. Cheng, R. J. Hyndman and A. Panagiotelis"} }
% \author{}
% \date{}

\maketitle

We would like to thank the reviewer for their thorough evaluation and constructive feedback on our manuscript. We have carefully considered all the comments and have made the necessary revisions to address them. Below, we provide detailed responses to each of the points raised by the reviewer and highlighted the responses in blue text.

\section*{General Comments}

\textit{It would be nice to include some short computation proving that the density estimator defined by formula (4) indeed integrates to one, if the \(H_{i}\) is the true metric matrix at point \(y_{i}\). Or give an "exact" counterpart of (4) that does integrate to one (is it (3)?)}

\textcolor{blue}{We appreciate the reviewer's positive feedback on our manuscript. In response to the suggestion, we clarify that Equation (3) serves as an ``exact'' counterpart of Equation (4). We now state:\\
	``The estimator has a similar structure to Equation (3) which can be seen as the counterpart to Equation (4) if the Riemannian is known exactly.''
}

\section*{Typos and Notational Consistency}

We are grateful for the reviewer's attention to details regarding typographical errors. We have addressed the following points:

1. \textit{p. 3 line 24: missing "metric"}

   \textcolor{blue}{%
   We have corrected the missing "metric" in the specified line. \\
   ``Critical to our estimator, is obtaining an estimate of $|\bm{H}(\bm{y})|^{1/2}$ for this purpose we use the Learn Metric algorithm of \textcite{Perrault-Joncas2013-pq} which augments any dimension reduction algorithm with an estimate of the Riemannian \textbf{metric} at each data point.''
   }

2. \textit{p. 6 line 50: it should be \(\exp _{p}(v)\) instead of \(\exp _{p}(q)\)}

   \textcolor{blue}{
   The notation has been corrected to \(\exp_{p}(v)\) as suggested. \\
   ``Consider the exponential map around \(\bm{p}\), given by \(exp_{\bm{p}}(\bm{v})\), mapping vectors in the tangent space, \(\bm{v}\in T_{\bm{p}}M\), to points on the manifold, \(\bm{q}\in M\).''
   }

3. \textit{p7, in reference "Brigant and Puechmorel", the name is Le Brigant and not Brigant}

   \textcolor{blue}{
   We have updated the reference in the bib file and the inline reference to "Le Brigant and Puechmorel" to reflect the correct name. \\
   ``For more on the volume density function, see \textcite{Gallot2004-rc} and \textcite{Le_Brigant2019-lj}.''}

4. \textit{In formula (1) the determinant seems to be denoted by \(|\cdot|\) while in formula (4) by det. Maybe homogenize the notations.}

   \textcolor{blue}{
   We have homogenized the notations for the determinant across the manuscript. The determinant is now consistently denoted by \(\det\) in all relevant formulas (1) and (4). \\
   \begin{equation}
    \label{eq:changevar}
    Pr(\bm{y}\in\psi(\mathcal{A}))=\int\limits_{\psi(\mathcal{A})} (f\circ\psi^{-1})(\bm{y})|\det \bm{H}(\bm{y})|^{1/2}d\bm{y},
    \end{equation}
    \begin{equation}
    \label{eq:denestimator}
    \hat{f}(\bm{y}_j) = \frac{1}{N} \sum_{i=1}^{N} \frac{1}{r^d} \bigg(\frac{|\det \bm{H}_j|}{|\det \bm{H}_i|} \bigg)^{1/2} K\bigg( \frac{\| \bm{H}^{-1/2}_i (\bm{y}_j - \bm{y}_i)\|}{r} \bigg).
    \tag{4}
    \end{equation}
    }

5. \textit{p. 11 line 54: Riemannian metric ?}

   \textcolor{blue}{
   We have reviewed and corrected the use of "Riemannian metric" as suggested. \\
   ``By treating the \(\bm{v}\) as an `output' embedding from input points \(\bm{x}\) that lie on the true manifold and applying the Learn Metric algorithm, we can obtain \(\bm{\Gamma}_i\) for \(i=1,\dots,n\) where \(\bm{\Gamma}_i\) is the Riemannian \textbf{metric} of the coordinate system given by \(\bm{v}\).''
   }


\vspace{0.5cm}

\textcolor{blue}{We believe these changes have strengthened the manuscript, and we are grateful for the reviewer's insights. We hope the revised version meets the standards for publication. Thank you once again for your valuable feedback.}


\printbibliography

% \bibliographystyle{plainnat}
% \bibliography{kderm}

\end{document}
